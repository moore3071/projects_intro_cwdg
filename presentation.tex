\documentclass{beamer}
\usetheme{Warsaw}
\usepackage{multicol}
\usepackage{hyperref}
\title{Let's build a snowman\\... or a website/webapp}
\date{}

\begin{document}
	\begin{frame}
		\maketitle
	\end{frame}

	\begin{frame}
		\frametitle{What languages are used on the web?}
		\begin{multicols}{2}
		\begin{itemize}
			\item \textbf{Ruby}
			\item \textbf{JavaScript}
			\item \textbf{Python}
			\item PHP
			\item Java
			\item C
			\item Lisp
			\item pretty much any language
		\end{itemize}
		\end{multicols}
		\begin{center}
			\Large \bf Just because the choice exists doesn't mean it's a good one
		\end{center}
	\end{frame}
	\note{Pretty much any language can be used. But some are better suited for the internet than others}

	\begin{frame}
		\frametitle{What frameworks are there?}
		\begin{multicols}{3}
		\begin{itemize}
			\item Ruby on Rails
			\item Meteor
			\item Angular
			\item React
			\item Grails
			\item And many more
		\end{itemize}
		\end{multicols}
	\end{frame}
	\note{This is a small list. Feel free to let the students suggest others for their project}

	\begin{frame}
		\frametitle{Static vs Dynamic (vs WebApps)}
		\begin{itemize}
			\item Static
				Can be compiled to simple HTML, CSS, and JS
			\item Dynamic
				Require interaction with the server to function and serve changing content (news sites, etc)
			\item WebApp
				Controversial, but focus primarily on letting the user perform actions (email interface, analytics, gdocs)
		\end{itemize}
	\end{frame}
	\note{Differentiate a static site, dynamic site, and web-app amicably. Note that the definition of webapp is controversial}

	\begin{frame}
		\frametitle{Static generators}
		\begin{multicols}{3}
			\begin{itemize}
				\item Jekyll
				\item Middleman
				\item Metalsmith
				\item Wintersmith
				\item Hyde
				\item \href{https://www.staticgen.com/}{So many more}
			\end{itemize}
		\end{multicols}
	\end{frame}
	\note{staticgen maintains a giant list of static site generators. All of these are either Ruby or JS}

	\begin{frame}
		\frametitle{Group up}
		Find a group that's either doing something that sounds interesting or are working with an interesting framework
	\end{frame}
	\note{Aim to end this iteration of projects by around November 1st. This gives padding before the end of the semester if they aren't even close}

\end{document}
